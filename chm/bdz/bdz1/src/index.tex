\maketitle
\tableofcontents
\newpage

\section{Первое задание}

\paragraph{Текст задания} ~\\
(а)Представить слагаемые и результат в виде нормализованного числа с плавающей точкой двойной точ-ности: $(-1)^{s}(2^{e-1023})(1.f)$, где $1.f$ записанов двоичном виде.(б) Если результат неточный (не умещается целиком в мантиссе), то указать относительную погрешность ошибки. Исходные данные в десятичной системе счисления.

 \[ -1593.5859375 \cdot 2^{128} + 1619.09765625 \cdot 2^{141}\]

\paragraph{Решение} ~\\
Для начала разберемся с первым числом $-1593.5859375 \cdot 2^{128}$:
\begin{align*}
 %первое выражение
  -1593.5859375_{10}& = 11000111001.1001011_{2} = 1.10001110011001011_{2} \cdot 2^{10} \\[1mm]
%второе выражение
  -1593.5859375_{10} \cdot 2^{128}& = (-1)^{1}(2^{1161-1023}) \cdot 1.10001110011001011
\end{align*}
$s = 1$, $e = 1023+128+10 = 1161$, $f = 1000111001100101\underbrace{0\ldots0_{\text{35 нулей}}}$\\[1em]
Теперь нормализуем $1619.09765625 \cdot 2^{141}$:
\begin{align*}
%первое выражение
  1619.09765625_{10}& = 11001010011.00011001_{2} = 1.100101001100011001_{2} \cdot 2^{10} \\[1mm]
%второе выражение
  1619.09765625 \cdot 2^{141}& = (-1)^{0}(2^{1174-1023})(1.100101001100011001)
\end{align*}
$s = 0$, $e = 1023 + 141 + 10 = 1174$, $f = 100101001100011001\underbrace{0\ldots0_{\text{34 нуля}}}$\\[1em]

\begin{multline*}
  11001010011.00011001 \cdot 2^{141}  - 11000111001.10001011 \cdot 2^{128} = \\[3mm]
  = 2^{141} (11001010011.00011001 - 11000111001.1001011 \cdot 2^{13}) = \\[3mm]
  = 2^{141} (11001010011.00011001 - 0,\underbrace{0\ldots0_{\text{12 нулей}}}110001110010001011) = \\[3mm]
  = 2^{141} (\underbrace{11001010011.00011001_{\text{19 битов}}}\underbrace{\empty_{\text{5 битов}}}\underbrace{\empty_{\text{18 битов}}})
\end{multline*}
Получается 42 бита $\Rightarrow$ число полностью поместится.

\section{Второе задание}

\paragraph{Текст задания} ~\\
Написать последовательность инструкций Matlab, формирующих указанную матрицу. Около каждой инструкции указать промежуточный результат в виде матрицы. Разрешается использовать матричные функции(eye, repmat, flipud и др.). Использовать циклы нельзя.\\[1em]
Входные данные: Целое $n \geq 20$\\[1em]
Надо получить:
\[
  \begin{pmatrix}
    0 & 1 & 0 & 2 & 0 & 3 & 0 & \cdots & n \\
    0 & 1 & 0 & 2 & 0 & 3 & 0 & \cdots & n \\
    \vdots & \vdots & \vdots & \vdots & \vdots & \vdots & \vdots & \ddots & \vdots \\
    0 & 1 & 0 & 2 & 0 & 3 & 0 & \cdots & n \\
    0 & 1 & 0 & 2 & 0 & 3 & 0 & \cdots & n \\
  \end{pmatrix}
  \} 2n
\]

\paragraph{Решение} ~\\
\begin{lstlisting}
  n = input();
  A \[= [1 : 0.5 : n]
  \end{lstlisting}
\[A = 1.00, 1.50, 2.00, 2.50, 3.00, 3.50, 4.00, \ldots, n\]

\begin{lstlisting}
  A = fix(A)
\end{lstlisting}
\[A = 1, 1, 2, 2, 3, 3, 4, 4, 5, 5, \ldots, n\]

\begin{lstlisting}
  A = [0, A]
\end{lstlisting}
\[A = (0, 1, 1, 2, 2, \ldots, n\]

\begin{lstlisting}
  A = repmat(A, 2*n, 1);
\end{lstlisting}
\[
  A =
  \begin{pmatrix}
    0 & 1 & 1 & 2 & 2 & 3 & 3 & \cdots n \\
    0 & 1 & 1 & 2 & 2 & 3 & 3 & \cdots n \\
    \vdots & \vdots & \vdots & \vdots & \vdots & \vdots & \vdots & \ddots & \vdots \\
    0 & 1 & 1 & 2 & 2 & 3 & 3 & \cdots n \\
  \end{pmatrix}
\]
