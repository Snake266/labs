\maketitle
\tableofcontents
\newpage
\section{(1) Первое задание}
Выберите некоторую функцию (например, $\sin x$, $\cos x$, $\exp x$, $\sh x$, $\ch x$, $\ln x$, \ldots), и некоторую точку $x$ из области определения функции. Найдите значение производной функции в выбранной точке (используя любую формулу численного дифференцирования) с точностью $10^{-3}$, $10^{-6}$. Пользоваться точным значением производной в качестве эталона запрещено.\\[5mm]

Для численного дифференцирования будем использовать формулу: $y^{'} \approx \frac{f(x) - f(x-h)}{h}$.

Посчитаем прозводную для $\sin x$, в точке $x = 0$:
\begin{lstlisting}
  >> (sin(0) - sin(0-10^-3))/10^-3
  ans = 0.999999833333342
  >> (sin(0) - sin(0-10^-6))/10^-6
  ans = 0.999999999999833
\end{lstlisting}

\section{(2) Второе задание}
Выберите некоторую функцию (например, $\sin x$, $\cos x$, $\exp x$, $\sh x$, $\ch x$, $\ln x$, \ldots) и некоторую точку x из области определения функции. Сравните погрешности у формул с разными порядками погрешностей (например, $y^{'} \approx \frac{y(x+h) - y(x)}{h}$ и $y^{'} \approx \frac{y(x+h) - y(x-h)}{2h}$) для последовательности убывающих шагов (например, $h = \frac{1}{2}, \frac{1}{4}, \frac{1}{8}$). С какими скоростями убывают погрешности для каждой формулы? Дайте теоретическую оценку и подтвердите ответ экспериментом.\\[5mm]

Сравнивать будем две формулы:$y^{'} \approx \frac{y(x+h) - y(x)}{h}$ и $y^{'} \approx \frac{y(x+h) - y(x-h)}{2h}$, с $h = \frac{1}{2}, \frac{1}{4}, \frac{1}{8}$, в точке $x = 0$, для функции $\sin x$

\begin{lstlisting}
  df1 = @(x, h) (sin(x+h) - sin(x))/h;
  df2 = @(x, h) (sin(x+h) - sin(x-h))/(2*h);

  x = 0;
  for h = [1/2, 1/4, 1/8]
    printf("Значение производной по первой формуле, для h = %f: %f\n", h, df1(x, h))
  end

  for h = [1/2, 1/4, 1/8]
    printf("Значение производной по второй формуле, для h = %f: %f\n", h, df2(x, h))
  end
\end{lstlisting}

\begin{lstlisting}[backgroundcolor=\color{cyan}]
  Значение производной по первой формуле, для h = 0.500000: 0.958851
  Значение производной по первой формуле, для h = 0.250000: 0.989616
  Значение производной по первой формуле, для h = 0.125000: 0.997398

  Значение производной по второй формуле, для h = 0.500000: 0.958851
  Значение производной по второй формуле, для h = 0.250000: 0.989616
  Значение производной по второй формуле, для h = 0.125000: 0.997398
\end{lstlisting}

\section{(3) Третье задание}
Неустойчивость численного дифференцирования. Выберите некоторую функцию (например, $\sin x$, $\cos x$, $\exp x$, $\sh x$, $\ch x$, $\ln x$, \ldots) и некоторую точку x из области определения функции. Попробуйте применить формулу $y^{'}(x) \approx \frac{y(y+h) - y(x)}{h}$ для стремящейся к нулю последовательности $h = \frac{1}{2}, \frac{1}{4}, \frac{1}{8}, \frac{1}{16}, \ldots$. Будет ли погрешность $\epsilon = \left |y^{'}(x) - \frac{y(x+h) - y(x)}{h} \right |$ монотонно убывать при уменьшении h? Сравните практический и теоритеческий результаты.\\[5mm]

\begin{lstlisting}
df = @(x, h) (sin(x+h) - sin(x))/h;

x = 0;
for h = [1/2, 1/4, 1/8, 1/16]
  dydx = df(x, h)
  epsilon = abs(cos(0) - dydx)
end
\end{lstlisting}

\begin{lstlisting}[backgroundcolor=\color{cyan}]
  dydx = 0.958851077208406
  epsilon = 4.114892279159399e-02

  dydx = 0.989615837018092
  epsilon = 1.038416298190825e-02

  dydx = 0.997397867081822
  epsilon = 2.602132918178457e-03

  dydx = 0.999349085478083
  epsilon = 6.509145219167900e-04
\end{lstlisting}
