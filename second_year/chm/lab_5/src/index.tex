\tableofcontents
\newpage
\section{(1) Первое задание}
Задайте функцию $f(x) = x^{3}$ на отрезке $[0, 1]$. Очевидно, определенный интеграл от функции $f(x)$ на этом отрезке равен $\frac{1}{4}$. Напишите программу, вычисляющую значение интеграла по формулам трапеций и Симпсона. Какую максимальную теоретическую ошибку мы при этом допускаем? Найдите реальные значение погрешности (абсолютное значение разности между теоретическим и аналитическим решением). Почему при вычислении интеграла по формуле Симпсона от данной функции ошибка равна нулю? Какие бы получились значения погрешностей для квадратичнойи линейной функций (предположите и проведите численный эксперимент для $f_{2}(x) = x^{2}$, $f_{1}(x) = \frac{x}{2}$ на отрезке $[0, 1]$).\\[2mm]

Формула трапеции:
\[
  \int_{a}^{b} f(x)dx \approx \sum_{i=1}^{n }h\frac{f(x_{i}) + f(x_{i-1})}{2}
\]
Погрешность для нее:
\[
  \frac{h^{2}(b-a)}{12}M_{2}\text{, где } M_{2} = \max_{[a,b]}|f^{''}(x)|
\]
Формула Симпсона:
\[
  \int_{a}^{b} f(x)dx \approx \frac{h}{6}(f_{i-1} + 4f_{i-1/2} + f_{i})
\]
Погрешность для неё:
\[
  |\Psi| \leq \frac{h^{2}(b-a)}{2880}M_{4} \text{, где } M_{4} = \max_{[a,b]}|f^{(4)}(x)|
\]
Функция для расчета по формуле трапеции:
\begin{lstlisting}
  function I = trapeze(f, x, h)
    I = 0;
    for i = 2 : length(x)
        I = I + (h/2)*(f(x(i)) + f(x(i-1)));
    end
  end
\end{lstlisting}

Функция для расчета по формуле Симпсона:
\begin{lstlisting}
  function I = simpson(f, x, h)
    I = 0;
    for i = 2 : length(x)
    I = I+(h/6)*(f(x(i-1))+4*f(x(i)-h/2)+f(x(i)));
    end
  end
\end{lstlisting}
Решение задачи:
\begin{lstlisting}
  f = @(x) x.^3; % |Функция|
  a = 0; b = 1; % |Границы|
  x = linspace(0, 1, 10); % |Промежуток|
  h = x(2) - x(1); % |Шаг разбиения|
  re = 1/4;
  M2_f = 6; %max abs(f''(x)), [x(i-1),x(i)]

  simpson(f, x, h)
  trapeze(f, x, h)
  teor_eps_trapeze = M2_f * (b - a) * h^2/12
  eps_simpson = abs(simpson(f, x, h) - re)
  eps_trapeze = abs(trapeze(f, x, h) - re)

  % Погрешности:
  Q = @(x)x.^2;
  L = @(x)x/2;
  disp('|Ошибка для квадратичной функции|:')
  abs(simpson(Q, x, h) - 1/3)
  disp('|Ошибка для линейной функции|:')
  abs(simpson(L, x, h) - 1/4)
\end{lstlisting}

Значение интеграла по формуле Симпсона: $I = 0.2500$\\
Значение интеграла по формуле трапеций: $I = 0.2531$\\
Теоретическая ошибка по формуле Симпсона: $0$\\
Теоретическая ошибка по формуле трапеций: $teor\_eps\_trapeze = 6.1728e-03$\\
Реальная ошибка по формуле Симпсона: $0$\\
Реальная ошибка по формуле трапеций: $eps\_trapeze = 3.0864e-03$\\
Ошибка для квадратичной функции:\\
$ans = 5.5511e-17$\\
Ошибка для линейной функции:\\
$ans = 2.7756e-17$\\

\section{(2) Второе задание}
Используя соотношение $\int_{0}^{1}\frac{1}{1+x^{2}}dx = arctg(1)$ найдите значение числа $\pi$ с точностью $10^{-6}$. В данном задании в процессе вычислений нельзя ис-пользовать встроенную константу \textbf{pi} для определения величины шага. Изкаких соображений выбирался шаг для получения указанной точности?

\section{(3) Третье задание}
Реализовать предыдущее задание, определяя точность методом Рунге. При численном вычислении интегралов последовательно с шагами $h$ и $\frac{h}{4}$ можно сократить число арифметических операций. Заметим, что приближенное значение интеграла $I_{\frac{h}{2}}$ есть сумма, часть слагаемых которойвозможно уже участвовало при вычислении $I_{h}$. Поэтому можно получить $I_{\frac{h}{2}}$ используя числовое значение $I_{h}$. то позволяет избежать повторногосуммирования части слагаемых.
