\tableofcontents
\newpage
\section{(1) Первое задание}
Задайте функцию $f(x) = x^{3}$ на отрезке $[0, 1]$. Очевидно, определенный интеграл от функции $f(x)$ на этом отрезке равен $\frac{1}{4}$. Напишите программу, вычисляющую значение интеграла по формулам трапеций и Симпсона. Какую максимальную теоретическую ошибку мы при этом допускаем? Найдите реальные значение погрешности (абсолютное значение разности между теоретическим и аналитическим решением). Почему при вычислении интеграла по формуле Симпсона от данной функции ошибка равна нулю? Какие бы получились значения погрешностей для квадратичнойи линейной функций (предположите и проведите численный эксперимент для $f_{2}(x) = x^{2}$, $f_{1}(x) = \frac{x}{2}$ на отрезке $[0, 1]$).\\[2mm]

\section{(2) Второе задание}
Используя соотношение $\int_{0}^{1}\frac{1}{1+x^{2}}dx = arctg(1)$ найдите значение числа $\pi$ с точностью $10^{-6}$. В данном задании в процессе вычислений нельзя ис-пользовать встроенную константу \textbf{pi} для определения величины шага. Изкаких соображений выбирался шаг для получения указанной точности?

\section{(3) Третье задание}
Реализовать предыдущее задание, определяя точность методом Рунге. При численном вычислении интегралов последовательно с шагами $h$ и $\frac{h}{4}$ можно сократить число арифметических операций. Заметим, что приближенное значение интеграла $I_{\frac{h}{2}}$ есть сумма, часть слагаемых которойвозможно уже участвовало при вычислении $I_{h}$. Поэтому можно получить $I_{\frac{h}{2}}$ используя числовое значение $I_{h}$. то позволяет избежать повторногосуммирования части слагаемых.
